\chapter*{Réduction d'échelle}
\section*{Introduction}
Ce programme a pour volonté de simuler dans un plan 2D, un système composé de $n$ corps soumis à la gravité. La programmation se fait en utilisant le code \textbf{Python} et \textbf{GitHub} (\verb!git@github.com:Florian-DELRIEU/Gravitationnal-Sytem.git!). Les différents corps seront donc en intéractions entres eux et chacun d'entre eux seront positionné de manière arbitraire avec une masse différentes.

\section*{Mise à l'échelle}
\subsection*{Constante de Kepler}
Le but de cette section est de trouver un nombre adimensionnel $\mu$ afin de réduire l'échelle d'un systeme gravitationnel tout en gardant une véracité dans la simulation. Considérons le système Soleil $S$ de masse $M$, et la terre $P$ de masse $m$, orbitant à une distance $R$ (en moyenne) avec une période $T$.
\begin{equation}
    \systeme{
    M = 1.99 \times 10^{30} \text{ kg} \\
    m = 5.97 \times 10^{24} \text{ kg} \\
    R = 1.5 \times 10^{8} \text{ km} \\
    T = 3.15 \times 10^{7} \text{ s} \\
    }
\end{equation}
Avec $G$ la constante de gravitation.
Je commence par écrire l'équation du mouvement induit par la gravité soit:
\begin{equation}
    m \dfrac{d^2 r}{dt^2} = - G \dfrac{M \cdot m}{r^2}
\end{equation}
avec $G$ étant la constante gravitationnelle. $m$ se simplifiant dans l'équation, Le système est composé de 4 variables et de trois dimensions, il peut alors être réduit à un seul paramètre adimensionnel. Il existe alors une fonction $\phi$ et une variable adimensionnelle $\Pi$ telle  que
\begin{equation}
    \phi ( \Pi ) = 0
\end{equation}
Afin de le rendre adimensionnel, il n'existe qu'une combinaison de ses variables telles que 
\begin{equation}
    [M]^a [R]^b [T]^c [G]^d = 0 
\end{equation}
On obtient alors assez simplement que :
\begin{equation}
\boxed{
    \Pi = G\dfrac{M T^2}{R^3}
    }
\end{equation}
On remarque alors que nous obtenons, a une constante près, la constante de Kepler à savoir:
\begin{equation}
    \dfrac{T^2}{a^3} = \dfrac{4 \pi^2}{GM} = \text{cte}
    \label{Eq:Kepler}
\end{equation}
Ce résultat montre alors que pour tout système gravitationnel on à $\Pi = 4 \pi^2$

\subsection*{Réduction d'échelle}
Avec la connaissance de la constante de Kepler (\ref{Eq:Kepler}), il est alors facile de réduire l'échelle des grandeurs afin de simplifier les variables dans le cadre d'une simulation. Si pour simplifier je pose $G = 1 \text{ SI}$, et $a = 1 \text{ SI}$ alors on obtient la relation
\begin{equation}
\boxed{
    M = \dfrac{4 \pi^2}{T^2} 
    }
\end{equation}
Il existe alors une relation simple entre la masse du parent et la période de rotation du satellite. Si je souhaite que mon satellite ait une période de révolution autour de 2 sec (en supposant que la vitesse soit bonne) alors, la masse du parent doit être de:
$$
M = \dfrac{4 pi^2}{4} = \pi^2 = 9.86
$$
Je vais alors prendre une masse M arrondi à $10$. La période de révolution sera alors légèrement plus rapide mais toujours de l'ordre de 2 sec ($T = 1.98 \text{ s}$).
\subsection*{Vitesse orbitale}
Un satellite est en orbite circulaire autour d'un satellite si son énergie cinétique $Ec$ est égale à son énergie potentielle de pesanteur $Epp$ (a condition de le vecteur vitesse soit perpendiculaire à la direction de la force de gravité).
\begin{equation}
    Ec = Epp
\end{equation}
$$
\dfrac{1}{2} m v^2 = m g r
$$$$
v^2 = m g r
$$
avec $g = GM/r$ l'accélération de la pesanteur. On obtient alors
\begin{equation}
\boxed{
    v = \sqrt{\dfrac{2 G M}{r}}
    }
\end{equation}
Si on se place dans le cadre de notre réduction d'échelle avec $r=a$, l'expression se réduit à:
\begin{equation}
    v = \sqrt{\dfrac{2 M}{a}} = \sqrt{20}
\end{equation}